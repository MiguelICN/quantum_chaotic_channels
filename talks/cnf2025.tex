\documentclass[10pt]{beamer}
\usetheme{metropolis}

% --- Essentials ---
\usepackage{appendixnumberbeamer}
\usepackage{booktabs}
\usepackage[scale=2]{ccicons}
\usepackage{pgfplots}
\usepgfplotslibrary{dateplot}
\usepackage{xspace}
\usepackage{graphicx}
\graphicspath{{figs/}}
\usepackage{physics}
\usepackage{amsmath}
%\usepackage{siunitx}

% --- Bibliography setup ---
\usepackage[%
    style=verbose,
    backend=bibtex,
    giveninits=true,      % Use initials for first names
    isbn=false, url=false, doi=false, eprint=false,
    maxcitenames=1,
    mincitenames=1
]{biblatex}

% Custom format: Author, Journal, vol, page, (year)
\DeclareFieldFormat[article]{journaltitle}{#1}
\DeclareFieldFormat[article]{volume}{\textbf{#1}} % Bold volume number
\DeclareFieldFormat[article]{pages}{#1} % Remove "p." for cleaner look

% Redefine the short citation format for verbose style
\renewbibmacro*{cite:short}{%
  \color{gray}\tiny
  \printtext[brackets]{%
    \printfield{journaltitle}%
    \setunit*{\space}%
    \printfield{volume}%
    \setunit*{\space}%
    \printfield{pages}%
    \setunit*{\space}%
    \printtext[parens]{\printfield{year}}%
  }
}

% Also modify the full citation format for consistency
\renewbibmacro*{cite:full}{%
  \color{gray}\tiny
  \printtext[brackets]{%
    \printfield{journaltitle}%
    \setunit*{\space}%
    \printfield{volume}%
    \setunit*{\space}%
    \printfield{pages}%
    \setunit*{\space}%
    \printtext[parens]{\printfield{year}}%
  }
}

%\addbibresource{references.bib}
\addbibresource{/home/jadeleon/Documents/doctorado/protocolo_candidatura/references.bib}


% --- Single, clean inline citation command ---
\newcommand{\scite}[1]{\textcolor{gray}{\scriptsize(\citeauthor{#1}, \citeyear{#1})}}

% --- Colors and theme styling ---
\definecolor{deepblue}{RGB}{46,78,126}
\definecolor{lightblue}{RGB}{200,220,255}
\setbeamercolor{palette primary}{bg=deepblue, fg=white}
\setbeamercolor{palette secondary}{bg=deepblue, fg=white}
\setbeamercolor{palette tertiary}{bg=deepblue, fg=white}
\setbeamercolor{structure}{fg=deepblue}

% --- Metadata ---
\title{Chaos through the lens of quantum channels}
%\subtitle{Brief Subtitle Explaining Domain}
\date{\today}
\author{\textbf{Jose Alfredo de Leon}\inst{1} \and 
Miguel Gonzalez\inst{2,3} \and
Carlos Diaz-Mejia\inst{2}}
\institute{
\inst{1}Instituto de Física, UNAM
\and 
\inst{2}Instituto de Ciencias Nucleares, UNAM
\and
\inst{3}Institute for Basic Science (IBS)
}
%\titlegraphic{\hfill\includegraphics[height=1cm]{logo.pdf}}

\begin{document}

\maketitle

% --- Slide 1 ---
\begin{frame}{Motivation \& Problem Statement}
\begin{columns}[T,onlytextwidth]
\column{0.5\textwidth}
  \textbf{Key Challenges:}
  \begin{itemize}
    \item Current limitation in field %\cite{landau2013course}
    \item Gap in existing literature %\cite{griffiths2005introduction}
    \item Unanswered scientific question\cite{pineda_2007_universal}
    \item Technological barrier\cite{deleon_2022_pauli} \cite{atas_2013_distribution}
  \end{itemize}
  
  \vspace{0.5cm}
  \textbf{Research Need:}
  \begin{itemize}
    \item Why this matters now
    \item Potential impact \scite{jackson1999classical}
  \end{itemize}

\column{0.5\textwidth}
  \begin{figure}
%        \includegraphics[width=\textwidth]{motivation_figure.pdf}
    \caption{Schematic showing the problem space \scite{taylor2005classical}}
  \end{figure}
\end{columns}
\end{frame}

% --- Slide 2 ---
\begin{frame}{Research Objectives}
  \begin{block}{Primary Objective}
    Development of novel method/technique/theory for...
  \end{block}
  
  \begin{exampleblock}{Specific Aims}
    \begin{enumerate}
      \item \textbf{Aim 1:} Characterize/develop/model...
      \item \textbf{Aim 2:} Implement/analyze/validate...
      \item \textbf{Aim 3:} Demonstrate/compare/optimize...
    \end{enumerate}
  \end{exampleblock}
  
  \vspace{0.5cm}
  Building upon previous work while addressing key limitations~\cite{atas_2013_distribution}.
\end{frame}

% --- Slide 3 ---
\begin{frame}{Theoretical Framework}
  \begin{theorem}[Key Theoretical Result]
    \[
    \mathcal{H} = \int_\Omega \left[\frac{1}{2}\kappa(\nabla\phi)^2 + f(\phi)\right] \dd{V}
    \]
    Based on Landau theory\footcite{landau2013course}
  \end{theorem}
  
  \begin{alertblock}{Fundamental Equation}
    \vspace{-0.2cm}
    \[
    \pdv{\psi}{t} = \mathcal{L}\psi + \mathcal{N}(\psi) + \eta(\vec{r},t)
    \]
    Extending previous formulations\footcite{goldstein2002classical}
  \end{alertblock}
  
  \begin{itemize}
    \item Mathematical foundation following\footcite{arfken2013mathematical}
    \item Key assumptions and their validity
    \item Novel theoretical contribution
  \end{itemize}
\end{frame}

% --- Slide 4 ---
\begin{frame}{Methodology}
  \begin{columns}[T,onlytextwidth]
    \column{0.6\textwidth}
      \textbf{Experimental/Numerical Setup:}
      \begin{itemize}
        \item Technique/equipment used\footcite{press2007numerical}
        \item Parameters and conditions
        \item Validation approach\footcite{allen2017computer}
        \item Statistical methods\footcite{sivia2006data}
      \end{itemize}
      
      \vspace{0.3cm}
      \textbf{Innovative Aspects:}
      \begin{itemize}
        \item New methodology developed
        \item Unique combination of techniques\footcite{frenkel2001understanding}
      \end{itemize}

    \column{0.4\textwidth}
      \begin{figure}
%        \includegraphics[width=\textwidth]{setup_diagram.pdf}
        \caption{Experimental/computational setup\footcite{landau2014guide}}
      \end{figure}
  \end{columns}
\end{frame}

% --- Slide 5 ---
\begin{frame}{Key Result 1: [Main Finding]}
  \begin{columns}[T,onlytextwidth]
    \column{0.6\textwidth}
      \begin{figure}
%        \includegraphics[width=\textwidth]{result1.pdf}
        \caption{Primary result showing [key finding]}
      \end{figure}
      
    \column{0.4\textwidth}
      \textbf{Observations:}
      \begin{itemize}
        \item Clear trend/pattern
        \item Statistical significance\footcite{sivia2006data}
        \item Comparison to expectation
      \end{itemize}
      
      \vspace{0.5cm}
      \textbf{Interpretation:}
      \begin{itemize}
        \item Physical/biological meaning
        \item Surprising/unexpected aspect\footcite{griffiths2005introduction}
      \end{itemize}
  \end{columns}
\end{frame}

% --- Motivation ----
\begin{frame}
Motivación con una figura
\end{frame}

% --- Quantum channels ----
\begin{frame}[t]{Canales cuánticos}
\begin{itemize}
\item Describen ruido cuántico, mediciones generalizadas y la \alert{dinámica 
de sistemas cuánticos abiertos}.
\item Mapeos CPTP: completamente positivos y que preservan la traza 
de la matriz de densidad. 
\begin{itemize}
\item CP: $\mathcal E \otimes I_k \geq 0 $
\item TP: $\Tr[\mathcal E(\rho)] = \Tr(\rho)$
\end{itemize}
\end{itemize}
\end{frame}

% --- Quantum chaos ----
\begin{frame}[t]{Caos cuántico}
Caos cuántico = RMT

\cite{atas_2013_distribution}
\cite{bohigas_1984_characterization}
\cite{berry_1997_level}

\includegraphics[width=0.5\textwidth]{rmt_goe.pdf}
\end{frame}

% --- Comparison with Wisniacki's results ----
\begin{frame}[t]{Previous}
\cite{mirkin_2021_quantum}~\cite{mirkin_2021_sensing}


\end{frame}

% --- Quantum chaos ----

% --- Quantum chaos ----

% --- References slide ---
\begin{frame}[allowframebreaks]{References}
  \printbibliography[heading=none]
\end{frame}

% --- Thank you slide ---
\begin{frame}[standout]
  \centering
  Thank You\\
  \vspace{1cm}
  \small Questions?\\
  \vspace{0.5cm}
  \scriptsize \texttt{your.email@institution.edu}\\
  \vspace{0.5cm}
  \tiny References available upon request
\end{frame}

\end{document}