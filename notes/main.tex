\documentclass[10pt,letterpaper]{article} 
\usepackage[margin=1in]{geometry}
\usepackage[utf8]{inputenc}
\usepackage[english,spanish]{babel}

\usepackage{amsmath}
\usepackage{amsfonts}
\usepackage{amssymb}
\usepackage{physics}
\usepackage{bbm}

\usepackage{graphicx}
\graphicspath{{figs/}}

\usepackage[dvipsnames]{xcolor}

\usepackage{float}

\newcommand{\eref}[1]{eq.~(\ref{#1})} 
\newcommand{\sref}[1]{sec.~\ref{#1}}
\newcommand{\fref}[1]{Fig.~\ref{#1}}
\newcommand{\tref}[1]{table~\ref{#1}}
\newcommand{\Eref}[1]{Eq.~(\ref{#1})} 
\newcommand{\Sref}[1]{Sec.~\ref{#1}}
\newcommand{\Fref}[1]{Fig.~\ref{#1}}  
\newcommand{\Tref}[1]{Table~\ref{#1}}

\usepackage{hyperref}
\hypersetup{
colorlinks=true,
linkcolor=blue,
filecolor=blue,      
citecolor=blue,
urlcolor=blue,
pdftitle={},
pdfauthor=author={Jose Alfredo de Leon},
}

\usepackage{tikz}

\usepackage[mathlines]{lineno}  \linenumbers \setlength\linenumbersep{5pt}
\usepackage[inline]{showlabels,rotating}
\renewcommand{\showlabelrefline}{\hrule width 0pt height 3ex depth 0pt}
\renewcommand{\showlabelfont}{\small\slshape\color{red!70}}
\usepackage[draft,inline,nomargin]{fixme} \fxsetup{theme=color}
\definecolor{jacolor}{RGB}{200,40,0} \FXRegisterAuthor{ja}{aja}{\color{jacolor}JA}
\FXRegisterAuthor{cp}{acp}{\color{blue}CP}
\FXRegisterAuthor{cd}{acd}{\color{purple}CD}

\decimalpoint

\newcommand{\one}{\mathbbm{1}}

\title{Quantum chaos and quantum channels}
\author{}
\date{\today}

\renewcommand{\labelenumii}{\arabic{enumi}.\arabic{enumii}}
\renewcommand{\labelenumiii}{\arabic{enumi}.\arabic{enumii}.\arabic{enumiii}}
\renewcommand{\labelenumiv}{\arabic{enumi}.\arabic{enumii}.\arabic{enumiii}.\arabic{enumiv}}

% --------------------------------------------------------------------------------------------------
% Comamands for this document
\newcommand{\mcE}{\mathcal E}
\newcommand{\mcP}{\mathcal P}
% --------------------------------------------------------------------------------------------------

\begin{document}
\maketitle

\section{Ideas}
The evolution of the chaometer qubit can be understood as a quantum channel:
\begin{align}
\rho_1(t) = 
\mcE[\rho_1(0)] = 
\Tr_E\qty(e^{-iHt} \rho_1(0)\otimes \rho_E(0) e^{-iHt}),
\end{align}
where $\rho_1(0) \otimes \rho_E(0) = \dyad{\psi(0)}$, 
with $\ket{\psi(0)}$ is a $L$-qubit random product state; and $H$ the spin chain 
Hamiltonian.

The quantum channel $\mcE$ can be written in its Kraus form:
\begin{align}
\mcE(\rho_1) = 
\sum_{i=1}^{r\leq 4} K_i \rho_1 K_j^\dagger,
\end{align}
with $r$ is the Kraus rank of $\mcE$.

The purity $\mcP$ of the chaometer now reads:
\begin{align}
\mcP [\mcE(\rho_1)] &=
\Tr \big\{\qty[\mcE(\rho_1)]^2 \big\}\\
&= \sum_{i,j=1}^{r\leq 4}
\Tr \qty( K_i \rho K_i^\dagger K_j \rho K_j^\dagger) \\
&= \sum_{i,j}^{r\leq 4}
\Tr \qty( K_j^\dagger K_i \rho K_i^\dagger K_j \rho).
\end{align}
My guess is that the chaotic information of the system has to be 
encoded in operators $K_i^\dagger K_j$. To further explore this idea 
I will examine the relationship $\mcP [\mcE(\rho_1)]  = 1 - \eta $.

Integrable:
\begin{align}
\mcP [\mcE(\rho_1)] = 
\min \qty[
\frac{1}{N} \sum_{i=1}^N 
\qty(\frac{1}{T}\int_0^T 
\Tr [\rho_i^2(t)] dt
)
] \approx 1
\end{align}

Chaotic:
\begin{align}
\mcP [\mcE(\rho_1)] = 
\max \qty[
\frac{1}{N} \sum_{i=1}^N 
\qty(\frac{1}{T}\int_0^T 
\Tr [\rho_i^2(t)] dt
)
] \approx \frac{1}{2}
\end{align}

\end{document}